%++++++++++++++++++++++++++++++++++++++++
\documentclass[letterpaper,12pt]{article}
\usepackage[ngerman,english]{babel}
\usepackage{siunitx} 
\usepackage{amsmath}  % improve math presentation
\usepackage{graphicx} % takes care of graphic including machinery
\usepackage[margin=1in,letterpaper]{geometry} % decreases margins
\setlength{\emergencystretch}{1em}
\usepackage{scrpage2}
\pagestyle{scrheadings}
\clearscrheadfoot
\ihead{Hand-Out}
\ohead{Hochschule Mannheim}
\ifoot{H.Merk}
\ofoot{18.12.2018}
%++++++++++++++++++++++++++++++++++++++++
\begin{document}	
		\section*{Hochfrequenzchirurgie}
		
		Unter der HF-Chirurgie versteht man den assistierenden Einsatz von elektrischer Energie in der Chirurgie zur thermisch induzierten Veränderung oder Zerstörung von Gewebezellen mit dem Ziel der Hämostase (Blutstillung), Gewebedurchtrennung oder -versiegelung.\\
		
		\textbf{Wesentliche Bauteile:}
		\begin{itemize}
			\item \emph{HF-Generator:} Sorgt für den benötigten Wechselstrom mit einer Frequenz von $f>\SI{300}{\kilo\hertz}$. Derart hohe Frequenzen sind nötig um faradische Effekte in den Muskeln und Nerven zu verhindern (Zucken, Herzrythmusstörung) und um die elektrische Energie effektiv auf das Gewebe zu übertragen.
			\item \emph{Applikator:} Applikatoren als Handwerkszeug und Stromspitze können je nach Anwendung in verschiedenen Bauformen auftreten. Hier kann ein isoliertes Handstück diverse Applikatorspitzen für den jeweiligen Einsatz aufnehmen.
			\item \emph{Erdung:} Die Erdung erfolgt mithilfe einer großflächigen Ableitelektrode oder einer Negativspitze (Pinzettenapplikator).
		\end{itemize}
		
		\paragraph{Gewebeveränderung:}
		Grundsätzlich kann das Gewebe verschiedene Formen, je nach thermischer Energieeinspeisung, annehmen. Hierbei sind die für die HF-Chirurgie zwei relevanten Formen die \emph{Denaturierung} (Koagulation, Blutstillung) und das \emph{Schneiden} (Verdampfen, Abtragen) des Gewebes. 
		
		Beim \emph{Verdampfen/Schneiden} wird ein sinusförmiger Wechselstrom generiert und kann je nach Schnittanforderung (sauberer Schnitt oder grober Schnitt mit Koagulation) modelliert werden. Hier kommen spitze Applikatoren zum Einsatz, welche eine hohe Stromdichte aufgrund der kleinen Kontaktfläche aufweisen. 
		
		Beim \emph{Denaturieren/Koagulieren} wird das Gewebe lediglich ausgetrocknet und versiegelt, weshalb ein großflächiger Applikator (z.B. Kugelaufsatz) zum Einsatz kommt um die Stromdichte zu senken. Der benötigte Strom wird durch eine Reihe diskontinuierlicher Wellenpakete erreicht, bei denen Ruhephasen die sinusförmigen Wellenzyklen unterbrechen und die Gewebebildung ein wenig abkühlen lassen.\\
		
		\textbf{Anwendung:}
		\begin{itemize}
			\item \emph{Monopolare Technik:} Der Strom fließt zwischen Applikatorelektrode und einer großflächigen Ableitgegenelektrode (z.B. an Oberschenkel oder Rücken des Patienten), wobei die Stromdichte an der Applikatorelektrode sehr hoch ist, sodass thermische Effekte auftreten. Die Stromdichte nimmt hin zur Aktiven Elektrode quadratisch zu, d.h. außerhalb des Wechselwirkungsvolumens fächern sich die E-Feldlinien derart auf, dass die Stromdichte für nicht relevante Körperstellen unkritisch ist.
			\item \emph{Bipolare Technik:} Bei der bipolaren Technik besitzt der Applikator zwei Spitzen, ähnlich einer Pinzette, zwischen denen die Spannung anliegt. Das zu bearbeitende Gewebe wird zwischen den Pinzettenspitzen genommen und stellt so einen Stromfluss durch das Gewebe her.
		\end{itemize}
				
\end{document}